\documentclass[12pt]{article}

\usepackage{graphicx}
\usepackage{paralist}
\usepackage{hyperref}
\usepackage{xspace}
\usepackage{amsfonts}
\usepackage{amsmath}

\newcommand{\latex}{\LaTeX\xspace}

\oddsidemargin 0mm
\evensidemargin 0mm
\textwidth 160mm
\textheight 200mm
\renewcommand\baselinestretch{1.0}

\pagestyle {plain}
\pagenumbering{arabic}

\newcounter{stepnum}

\title{Assignment 2}
\author{COMP SCI 2ME3 and SFWR ENG 2AA4}

\begin {document}

\maketitle

\section{Dates and Deadlines}

\begin{description}
\item [Assigned:] January 29, 2019
\item [Part 1:]  February 11, 2019
\item [Receive Partner Files:] February 16, 2019
\item [Part 2:] February 18, 2019
\item [Last Revised:] \today
\end{description}

\noindent All submissions are made through git, using your own repo located at:\\

\texttt{https://gitlab.cas.mcmaster.ca/se2aa4\_cs2me3\_assignments\_2018/[macid].git}\\

\noindent where \texttt{[macid]} should be replaced with your actual macid.  The
time for all deadlines is 11:59 pm.  If you notice problems in your Part 1 {\tt
  *.py} files after the deadline, you should fix the problems and discuss them
in your Part 2 report.  However, the code files submitted for the Part 1
deadline will be the ones graded.

\section{Introduction}

The purpose of this software design exercise is to write a Python program that
follows the given formal specification.  The given specification covers a
problem similar to A1 - the allocation of first year engineering students into
their respective second year programs.  As for the previous assignment, you will
use doxygen, make, LaTeX and Python (version 3).  In addition, this assignment
will use pytest for unit testing and flake8 to verify that its
pep8-inspired standard is enforced.  This assignment also takes
advantage of functional programming in Python.

All of your code, except for the testing files, should be documented using
doxygen.  Using doxygen on the testing files is optional.  Your report should be
written using \latex.  Your code should follow the given specification exactly.
In particular, you should not add public methods or procedures that are not
specified and you should not change the number or order of parameters for
methods or procedures.  If you need private methods or procedures, please use
the Python convention of naming the files with the double underscore
(\texttt{\_\_methodName\_\_}) (dunders).  \textbf{Please follow specified naming
  EXACTLY.  You do not want to lose marks for a simple mistake.}

For the purpose of understandability, the specification provided at the end of
the assignment uses notation that is slightly different from the Hoffman and
Strooper notation.  Specifically the types for real and natural numbers are
represented by $\mathbb{R}$ and $\mathbb{N}$, respectively.  (In this
specification, the natural numbers are assumed to include 0.)  Booleans are
represented by $\mathbb{B}$.  Also, subscripts are used for indexing a sequence.
For instance, $x_i$ means the same thing as $x[i]$.  A subsection has also been
added to the template for local types.  The purpose of these local types is for
specification; they are not exported.

\subsection{Installing flake8}

We will use flake8 to ensure your Python code meets the style conventions of
the course. You will need to install two `pip' packages. This is the standard
way to install packages for Python.

First run the following command in your terminal:

\begin{verbatim}
pip --version
\end{verbatim}

If the output includes `Python 3,' then run the following instructions:

\begin{verbatim}
pip install flake8
pip install pep8-naming
\end{verbatim}

\textbf{If it does not,} then use \texttt{pip3} for the following alternate instructions.

\begin{verbatim}
pip3 install flake8
pip3 install pep8-naming
\end{verbatim}

\subsection{Running flake8}

Run the following command in your A2 directory.  This will inform you of the
location and types of style violations.  You can find more information on what
each error means here: https://lintlyci.github.io/Flake8Rules/

\begin{verbatim}
flake8
\end{verbatim}

\section*{Part 1}

\section *{Step \refstepcounter{stepnum} \thestepnum}

Write the modules \texttt{StdntAllocTypes.py}, \texttt{SeqADT.py},
\texttt{DCapALst.py}, \texttt{AALst.py}, \texttt{SALst.py} and \texttt{Read.py}
following the specification given at the end of the assignment.

\section *{Step \refstepcounter{stepnum} \thestepnum}

Write a module (named {\tt test\_All.py}), using pytest, that tests the
following modules: {\tt SeqADT.py}, {\tt DCapALst.py} and {\tt SALst.py}.  The
given makefile {\tt Makefile} has a rule \texttt{test} for running your tests.
Each procedure/method should have at least one test case.  Record your rationale for
test case selection and the results of using this module to test the procedures
in your modules.  (You will submit your rationale with your report in
Step~\ref{StepReport}.)  Please make an effort to test normal cases, boundary
cases, and exception cases.  Your test program should compare the calculated
output to the expected output and provide a summary of the number of test case
that have passed or failed.

\section *{Step \refstepcounter{stepnum} \thestepnum}

Test the supplied \texttt{Makefile} rule for {\tt doc}.  This rule should
compile your documentation into an html and \latex version.  Along with the
supplied \texttt{Makefile}, a doxygen configuration file is also given in your
initial repo.  You should not have to change these files.

\section *{Step \refstepcounter{stepnum} \thestepnum}

Submit (add, commit and push) all Python files using git.  (Of course, you will
be doing this throughout the development process.  This step is to explicitly
remind you that the version that will be graded is the one we see in the repo.)
Please \textbf{do not change} the names and locations for the files already
given in your git project repo.  You should also push any input data files you
created for testing purposes.  For Part 1, the only files that you should modify
are the Python files and the only ``new'' files you should create are the input
data files.  Changing other files could result in a serious grading penalty,
since the TAs might not be able to run your code and documentation generation.
You should NOT submit your generated documentation (html and latex folders).  In
general, files that can be regenerated are not put under version control.

Optionally, you can tag your final submission of Part 1 of the assignment with
the label \texttt{A2Part1}.

\section*{Part 2}

Your {\tt SeqADT.py}, {\tt DCapALst.py} and {\tt SALst.py} files will
automatically be pushed to your partner's repo and vice versa.
\textbf{Including your name in your partner code files is optional.} 

\section *{Step \refstepcounter{stepnum} \thestepnum}

After you have received your partner's files, replace your corresponding files
with your partner's.  Do not initially make any modifications to any of the
code.  Run your test module and record the results.  Your evaluation for this
step does not depend on the quality of your partner's code, but only on your
discussion of the testing results.  If the tests fail, for the purposes of
understanding what happened, you are allowed to modify your partner's code.

\section *{Step \refstepcounter{stepnum} \thestepnum \label{StepReport}} 

Write a report using \latex (\texttt{report.tex}) following the template given
in your repo.  Optionally, the final submission can have the tag
\texttt{A2Part2}.  The report should include the following:

\begin{enumerate}
\item Your name and macid.
\item Your updated Python files.
\item Your partner's files.
\item The results of testing your files (along with the rational for test case
  selection).  The summary of the results should consist of the following: the
  number of passed and failed test cases, and brief details on any failed test
  cases.
\item The results of testing your files combined with your partner's files.
\item A discussion of the test results and what you learned doing the exercise.
  List any problems you found with
\begin{inparaenum} 
\item your program,
\item your partner's module, and
\end{inparaenum}
\item The specification of this assignment imposed design decisions on you.
  Please provide a critiques of the design.  What did you like?  What areas need
  improvement?  How would you propose changing the design?
\item Answers to the following questions
\begin{enumerate}[a)]
\item Contrast the natural language of A1 to the formal specifiation of A2.
  What are the advantages and disadvantages of each approach?
\item The specification makes the assumption that the gpa will be between 0 and
  12.  How would you modify the specification to change this assumption to an
  exception?  Would you need to modify the specification to replace a record
  type with a new ADT?
\item If we ignore the functions \texttt{sort}, \texttt{average} and
  \texttt{allocate}, the two modules \texttt{SALst} and \texttt{DCapALst} are
  very similar.  Ignoring the functions mentioned, how could the documentation
  be modified to take advantage of the similarities?
\item A1 had a question about generality of an interface.  In what ways is A2
  more general than A1?
\item The list of choices for each student is represented by a custom object,
  SeqADT, instead of a Python list.  For this specific usage, what are the
  advantages of using SeqADT over a regular list?
\item Many of the strings in A1 have been replaced by enums in A2.  For these
  cases, what advantages do enums provide?  Why weren't enums also introduced in
  the specification for macids?  
\end{enumerate}
\end{enumerate}

The writing style for the report should be professional, but writing in the
first person is fine.  Some of your ideas can be summarized in lists, but most
of the report should be written in full sentences.  Spelling and grammar is
important.

Commit and push \texttt {report.tex} and \texttt{report.pdf}.  Although the pdf
file is a generated file, for the purpose of helping the TAs, we'll make an
exception to the general rule of avoiding version control for generated files.
If you have made any changes to your Python files, you should also push those
changes.

\subsubsection*{Notes}

\begin{enumerate}
\item Your git repo will be organized with the following directories at the
  top level: {\tt A1}, {\tt A2}, {\tt A3}, and {\tt A4}. 
\item Inside the {\tt A2} folder you will start with initial stubs of the files
  and folders that you need to use.  Please do not change the names or locations
  of any of these files or folders.
\item Please put your name and macid at the top of each of your source
  files, except for those that you share with a partner.  Including your name
  and macid is optional for those files.
\item Your program must work on mills when compiled with its versions of Python
  (version 3), LaTeX, doxygen, make, pytest, pytest-cov, and flake8.
\item Python specifics:
\begin{itemize}
\item The exceptions in the specification are intentionally selected to use the
  names of existing Python exceptions.  There is no need to define new
  exceptions.
\item Although efficient use of computing resources is always a good goal, your
  implementation will be judged on correctness and not on performance.
\item For the Python implementation of an abstract module, use
  \texttt{@staticmethod}.  Your access programs should be called via the module
  name followed by a dot and then the access program name.  For instance, for a
  module named \texttt{Data}, the access program is accessed using
  \texttt{Data.accessProg}, not simply \texttt{accessProg} or
  \texttt{Data\_accessProg}.  The call \texttt{Data.Data\_accessProg} is also
  incorrect.
\item For types that are defined as a set of potential values (like GenT), you
  should use an enumerated type
  (\href{https://docs.python.org/3/library/enum.html}
  {https://docs.python.org/3/library/enum.html}).
\item For types that are simply records, use the \texttt{typing.NamedTuple}
  class from
  \href{https://docs.python.org/3/library/typing.html#typing.NamedTuple}
  {https://docs.python.org/3/library/typing.html\#typing.NamedTuple}.
\item Since the specification is silent on this point, for methods that return
  an object, or use objects in their state, you can decide to either use
  references or construct new objects.  The implementation will be easier if you
  just work in terms of references to objects.
\item A sample program (\texttt{A2Examples.py}) that uses the modules in this
  specification is available in the repo.  You can use this to do an initial
  test that your interface matches the specification.
\item Sample files (\texttt{StdntData.txt} and \texttt{DeptCap.txt}) are
  available in the repo.
\item Marking scheme will be available in the course repo.  The marking scheme
  will include some grades based on git usage, make, unit testing, and flake8.
\end{itemize}
\item \textbf{Your grade will be based to a significant extent on the ability of
    your code to compile and its correctness.  If your code does not compile,
    then your grade will be significantly reduced.}
\item \textbf{Any changes to the assignment specification will be announced in
    class.  It is your responsibility to be aware of these changes.  Please
    monitor all pushes to the course git repo.}
\end{enumerate}

\newpage

\section* {Student Allocation Types Module}

\subsection*{Module}

StdntAllocTypes

\subsection* {Uses}

SeqADT(T)

\subsection* {Syntax}

\subsubsection* {Exported Constants}

None

\subsubsection* {Exported Types}

GenT = \{male, female\}\\ 
DeptT = \{civil, chemical, electrical, mechanical, software, materials, engphys\}\\
~\\
SInfoT = tuple of (fname: string, lname: string, gender: GenT, gpa:
$\mathbb{R}$, choices: SeqADT(DeptT), freechoice: $\mathbb{B}$)

\subsubsection* {Exported Access Programs}

None

\subsection* {Semantics}

\subsubsection* {State Variables}

None

\subsubsection* {State Invariant}

None

\subsubsection* {Assumptions}

For SInfoT the gpa will always lie between 0 and 12.0.

\newpage

\section* {Sequence ADT Module}

\subsection*{Generic Template Module}

SeqADT(T)

\subsection* {Uses}

None

\subsection* {Syntax}

\subsubsection* {Exported Constants}

None

\subsubsection* {Exported Types}

SeqADT = ?

\subsubsection* {Exported Access Programs}

\begin{tabular}{| l | l | l | p{5cm} |}
\hline
\textbf{Routine name} & \textbf{In} & \textbf{Out} & \textbf{Exceptions}\\
\hline
new SeqADT & sequence of T & SeqADT & ~\\
\hline
start & ~ & ~ & ~\\
\hline
next & ~ & T & StopIteration\\
\hline
end & ~ & $\mathbb{B}$ & ~\\
\hline
\end{tabular}

\subsection* {Semantics}

\subsubsection* {State Variables}

$s$: sequence of T\\
$i$: integer

\subsubsection* {State Invariant}

$i \in [0..|s|]$

\subsubsection* {Assumptions}

The constructor for SeqADT will not be called with an empty sequence as input.

\subsubsection* {Access Routine Semantics}

\noindent new SeqADT($x$):
\begin{itemize}
\item transition: $s, i := x, 0$
\item output: $out := \mbox{self}$
\item exception: none
\end{itemize}

\noindent start():
\begin{itemize}
\item transition: $i := 0$
\item exception: none
\end{itemize}

\noindent next():
\begin{itemize}
\item transition-output: $i, out := i + 1, s[i]$
\item exception: $( i \geq (|s|) \Rightarrow \text{StopIteration})$
\end{itemize}

\noindent end():
\begin{itemize}
\item output: $out := i \geq |s|$
\item exception: None
\end{itemize}

\newpage

\section* {Department Capacity Association List Module}

\subsection*{Module}

DCapALst

\subsection* {Uses}

StdntAllocTypes

\subsection* {Syntax}

\subsubsection* {Exported Constants}

None

\subsubsection* {Exported Types}

None

\subsubsection* {Exported Access Programs}

\begin{tabular}{| l | l | l | p{5cm} |}
\hline
\textbf{Routine name} & \textbf{In} & \textbf{Out} & \textbf{Exceptions}\\
\hline
init & ~ & ~ & ~\\
\hline
add & DeptT, $\mathbb{N}$ & ~ & KeyError\\
\hline
remove & DeptT & ~ & KeyError\\
\hline
elm & DeptT & $\mathbb{B}$ & ~\\
\hline
capacity & DeptT & $\mathbb{N}$ & KeyError\\
\hline
\end{tabular}

\subsection* {Semantics}

\subsubsection* {State Variables}

$s$: set of tuple of (dept: DeptT, cap: $\mathbb{N}$)

\subsubsection* {State Invariant}

None

\subsubsection* {Assumptions}

DCapALst.init() is called before any other access program.

\subsubsection* {Access Routine Semantics}

\noindent init():
\begin{itemize}
\item transition: $s := \{ \}$
\item exception: none
\end{itemize}

\noindent add($d$, $n$):
\begin{itemize}
\item transition: $s := s \cup \{ \langle d, n \rangle \}$
\item exception: $(\langle d, ? \rangle \in s \Rightarrow \text{KeyError} )$
  (? can be any natural number)
\end{itemize}

\noindent remove($d$):
\begin{itemize}
\item transition: $s := s - \{ \langle d, n \rangle \}$ where $\langle d, n
  \rangle \in s$
\item exception: $(\langle d, n \rangle \notin s \Rightarrow \text{KeyError} )$
\end{itemize}

\noindent elm($d$):
\begin{itemize}
\item output: $out := \langle d, n \rangle \in s$
\item exception: none
\end{itemize}

\noindent capacity($d$):
\begin{itemize}
\item output: $out := n$ where $\langle d, n \rangle \in s$
\item exception: $(\langle d, n \rangle \notin s \Rightarrow \text{KeyError} )$
\end{itemize}

\newpage

\section* {Allocation Association List Module}

\subsection*{Module}

AALst

\subsection* {Uses}

StdntAllocTypes

\subsection* {Syntax}

\subsubsection* {Exported Constants}

None

\subsubsection* {Exported Types}

None

\subsubsection* {Exported Access Programs}

\begin{tabular}{| l | l | l | p{5cm} |}
\hline
\textbf{Routine name} & \textbf{In} & \textbf{Out} & \textbf{Exceptions}\\
\hline
init & ~ & ~ & ~\\
\hline
add\_stdnt & DeptT, string & ~ & ~\\
\hline
lst\_alloc & DeptT & sequence of string & ~\\
\hline
num\_alloc & DeptT & $\mathbb{N}$ & ~\\
\hline
\end{tabular}

\subsection* {Semantics}

\subsubsection* {State Variables}

$s$: set of AllocAssocListT

\subsubsection* {State Invariant}

None

\subsubsection* {Assumptions}

AALst.init() is called before any other access program.

\subsubsection* {Access Routine Semantics}

\noindent init():
\begin{itemize}
\item transition: $s := \{ d: \text{DeptT} | d \in \text{DeptT} : \langle d, []
  \rangle \}$
\item exception: none
\end{itemize}

\noindent add\_stdnt($\mathit{dep}$, $m$):
\begin{itemize}
\item transition: $s := \{ \langle d, L \rangle : \text{AllocAssocListT} |
  \langle d, L \rangle \in s : (d = \mathit{dep} \Rightarrow \langle d, L || [m] \rangle
   | \text{True} \Rightarrow \langle d, L \rangle) \}$
\item exception: none
\end{itemize}

\noindent lst\_alloc($d$):
\begin{itemize}
\item output: $out := L$ where $\langle d, L \rangle \in s$
\item exception: none
\end{itemize}

\noindent num\_alloc($d$):
\begin{itemize}
\item output: $out := |L|$ where $\langle d, L \rangle \in s$
\item exception: none
\end{itemize}

\subsection*{Local Types}

AllocAssocListT = tuple of (dept: DeptT, sequence of string)

\newpage

\section* {Student Association List Module}

\subsection*{Module}

SALst

\subsection* {Uses}

StdntAllocTypes\\
AALst\\
DCapALst

\subsection* {Syntax}

\subsubsection* {Exported Constants}

None

\subsubsection* {Exported Types}

None

\subsubsection* {Exported Access Programs}

\begin{tabular}{| l | l | l | p{5cm} |}
\hline
\textbf{Routine name} & \textbf{In} & \textbf{Out} & \textbf{Exceptions}\\
\hline
init & ~ & ~ & ~\\
\hline
add & string, SInfoT & ~ & KeyError\\
\hline
remove & string & ~ & KeyError\\
\hline
elm & string & $\mathbb{B}$ & ~\\
\hline
info & string & SInfoT & KeyError\\
\hline
sort & $\text{SInfoT} \rightarrow \mathbb{B}$ & sequence of string & ~\\
\hline
average & $\text{SInfoT} \rightarrow \mathbb{B}$ & $\mathbb{R}$ & ValueError\\
\hline
allocate & ~ & ~ & RuntimeError\\
\hline

\end{tabular}

\subsection* {Semantics}

\subsubsection* {State Variables}

$s$: set of StudentT

\subsubsection* {State Invariant}

None

\subsubsection* {Assumptions}

SALst.init() is called before any other access program.
DCapALst has been fully populated for all departments before running
allocate.  The following assumptions apply to the data:
\begin{itemize}
	\item The free choice students will never number so many that they will
          fill a department.
        \item The case will never arise where the next student to be added to a
          department will exceed the capacity of the department, but have the
          exact same gpa as the last student allocated to that department,
\end{itemize}

\subsubsection* {Access Routine Semantics}

\noindent init():
\begin{itemize}
\item transition: $s := \{ \}$
\item exception: none
\end{itemize}

\noindent add($m$, $i$):
\begin{itemize}
\item transition: $s := s \cup \{ \langle m, i \rangle \}$
\item exception: $(\langle m, ? \rangle \in s \Rightarrow \text{KeyError} )$
\end{itemize}

\noindent remove($m$):
\begin{itemize}
\item transition: $s := s - \{ \langle m, i \rangle \}$ where $\langle m, i
  \rangle \in s$
\item exception: $(\langle m, i \rangle \notin s \Rightarrow \text{KeyError} )$
\end{itemize}

\noindent elm($m$):
\begin{itemize}
\item output: $out := \langle m, i \rangle \in s$
\item exception: none
\end{itemize}

\noindent info($m$):
\begin{itemize}
\item output: $out := i$ where $\langle m, i \rangle \in s$
\item exception: $(\langle m, i \rangle \notin s \Rightarrow \text{KeyError} )$
\end{itemize}

\noindent sort($f$):
\begin{itemize}
\item output: 
$out := L: \text{sequence of string}$, such that\\
$(\forall \langle m, i \rangle : \text{StudentT} |
  \langle m, i \rangle \in s \wedge f(i) : (\exists j: \mathbb{N} | j \in [0 .. |s| - 1] :
  L_j = m)) \wedge (\forall k:\mathbb{N} | k \in [0..|L|-2] :
  \text{get\_gpa}(L_k, s) \geq \text{get\_gpa}(L_{k+1}, s))$
\item exception: none
\end{itemize}

\noindent average($f$):
\begin{itemize}
\item output: $$out := \frac{(+ i : \text{SInfoT} | i \in \mathit{fset}: i.\text{gpa})}{|\mathit{fset}|}
  \text{ where }
  \mathit{fset} = \{ \langle m, i \rangle : \text{StudentT} | \langle m, i
  \rangle \in s \wedge f(i): i \}$$
\item exception: $(\{ \langle m, i \rangle : \text{StudentT} | \langle m, i
  \rangle \in s \wedge f(i): i \} = \emptyset \Rightarrow \text{ValueError})$
\end{itemize}

\noindent allocate():
\begin{itemize}
\item transition: \textit{\# procedural specification}\\
$\text{AALst.init()}$\\
$F = \text{SALst.sort}(\lambda t \rightarrow t.\text{freechoice} \wedge
t.\text{gpa} \geq 4.0)$\\
for all $m$ in $F$\\
$~~~~~ch = \text{SALst.info}(m).\text{choices}$\\
$~~~~~\text{AALst.add\_stdnt}(ch.\text{next}(), m)$\\
~\\
$S = \text{SALst.sort}(\lambda t \rightarrow \neg t.\text{freechoice} \wedge
t.\text{gpa} \geq 4.0)$\\
for all $m$ in $S$\\
$~~~~~ch = \text{SALst.info}(m).\text{choices}$\\
$~~~~~\mathit{alloc} = \text{False}$\\
$~~~~~\text{while } \neg \mathit{alloc} \wedge \neg ch.\text{end}()$\\
$~~~~~~~~d = ch.\text{next}()$\\
$~~~~~~~~\text{if AALst.num\_alloc}(d) < \text{DCapALst.capacity}(d)$\\
$~~~~~~~~~~\text{AALst.add\_stdnt}(d, m)$\\
$~~~~~~~~~~\mathit{alloc} = \text{True}$\\
$~~~~~\text{if } \neg \mathit{alloc} \text{ raise(RuntimeError)}$

\item exception: none
\end{itemize}

\subsection*{Local Types}

StudentT = tuple of (macid: string, info: SInfoT)

\subsection*{Local Functions}

get\_gpa: $\text{string} \times \text{set of StudentT}$\\

\noindent get\_gpa($m$, $s$) $\equiv i.\text{gpa} \text{ for } \langle m, i \rangle
\in s $\\

\newpage

\section* {Read Module}

\subsection* {Module}

Read

\subsection* {Uses}

StdntAllocTypes, DCapALst, SALst

\subsection* {Syntax}

\subsubsection* {Exported Constants}

None

\subsubsection* {Exported Access Programs}

\begin{tabular}{| l | l | l | l |}
\hline
\textbf{Routine name} & \textbf{In} & \textbf{Out} & \textbf{Exceptions}\\
\hline
load\_stdnt\_data & $s: \mbox{string}$ & ~ & ~\\
\hline
load\_dcap\_data & $s: \mbox{string}$ & ~ & ~\\
\hline
\end{tabular}

\subsection* {Semantics}

\subsubsection* {Environment Variables}

stdnt\_data: File listing student data\\
dept\_capacity: File listing department capacities

\subsubsection* {State Variables}

None

\subsubsection* {State Invariant}

None

\subsubsection* {Assumptions}

The input file will match the given specification.

\subsubsection* {Access Routine Semantics}

\noindent load\_stdnt\_data($s$)
\begin{itemize}
\item transition: read data from the file stdnt\_data associated with the string s.
  Use this data to update the state of the SALst module.  Load will first
  initialize SALst (SALst.init()) before populating SALst with student data that
  follows the types in StdntAllocTypes.

  The text file has the following format, where $id_i$, $fn_i$, $ln_i$, $g_i$,
  $gpa_i$, $[ch_i^0, ch_i^1, ..., ch_i^{n-1}]$ and $fc_i$ stand for strings that
  represent the ith student's macid, first name, last name, gender, grade point
  average, list of choices and free choice, respectively.  The gender is
  represented by either the string ``male'' or ``female.''  The list of choices
  comes from strings following the department names in the type DeptT.  The list
  of choices has length $n$.  $fc_i$ is either the string ``True'' or the string
  ``False.''  All data values in a row are separated by commas.  Rows are
  separated by a new line.  The data shown below is for a total of $m$ students.

  \begin{equation}
    \begin{array}{ccccccc}
      id_0, & fn_0, & ln_0, & g_0, & gpa_0, & [ch_0^0, ch_0^1, ..., ch_0^{n-1}], & fc_0 \\
      id_1, & fn_1, & ln_1, & g_1, & gpa_1, & [ch_1^0, ch_1^1, ..., ch_1^{n-1}], & fc_1\\
      id_2, & fn_2, & ln_2, & g_2, & gpa_2, & [ch_2^0, ch_2^1, ..., ch_2^{n-1}], & fc_2
      \\
      ..., & ..., & ..., & ..., & ..., & [..., ..., ], & ...
      \\
      id_{m-1}, & fn_{m-1}, & ln_{m-1}, & g_{m-1}, & gpa_{m-1}, & [ch_{m-1}^0, ch_{m-1}^1, ..., ch_{m-1}^{n-1}], & fc_{m-1} \\
    \end{array}
  \end{equation}

\item exception: none
\end{itemize}

\noindent load\_dcap\_data ($s$)
\begin{itemize}
\item transition: read data from the file dept\_capacity associated with the string s.
  Use this data to update the state of the DCapALst module.  Load will first
  initialize DCapALst (DCapALst.init()) before populating DCapALst with
  department capacity data.

  The text file has the following format.  Each department is identified by a
  string with the department name, and then a string for the natural number that
  represents the department's capacity.  All data values in a row are
  separated by commas.  Rows are separated by a new line.

  \begin{tabular}{ll}
    civil, & $n_\text{civil}$\\
    chemical, & $n_\text{chemical}$\\
    electrical, & $n_\text{electrical}$\\
    mechanical, & $n_\text{mechanical}$\\
    software, & $n_\text{software}$\\
    materials, & $n_\text{materials}$\\
    engphys, & $n_\text{engphys}$\\
  \end{tabular}

\item exception: none
\end{itemize}

\end {document}
